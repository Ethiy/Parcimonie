% Author : Oussama ENNAFII
% MVA Master, ENS Cachan

\documentclass[english]{beamer}
\usepackage[utf8]{inputenc}
\usepackage[T1]{fontenc}
\usepackage{amsmath,amssymb,MnSymbol,amsbsy}
\usepackage{graphicx}
\usetheme{Berlin}
\usepackage{float}


\title{Finding structure with randomness - Probabilistic Algorithms for Constructing Approximate Matrix Decomposition}
\author{Oussama \textsc{Ennafii}}
\institute[ENS]{
\structure{
Ecole Normale Superieur de Cachan}
}

\begin{document}
\frame{\maketitle}

\frame{
\frametitle{Problematic:}
\begin{block}{Issue to solve:}
Traditional low rank approximation algorithms such as the QR decomposition and SVD  are not addapted to very huge or inaccurate matrices.
\end{block}
\ \\
We need to find a framework to apply to these kinds of reccurent problems.
}

\frame{
\frametitle{Two stage approximation framework:}
\begin{block}{Two stage approximation framework}
\begin{itemize}
\item{Stage A}: Compute an orthornormal low rank basis $\bold{Q}$ such that, \\
$\bold{A} \approx \bold{Q}\bold{Q}^* \bold{A}$
\item{Stage B}: Compute the matrix factorisation on $\bold{B} = \bold{Q}^T
\bold{A}$
\end{itemize}
\end{block}

The randomization, in Stage A, permits to span the range of $A$ very efficiently.
}

\frame{
\frametitle{Problem formulation:}
\begin{block}{The fixed precision problem}
Given $\bold{A}$ and $\epsilon$, find $\bold{Q}$ s.t.
$$\lVert \bold{A} - \bold{Q}\bold{Q}^T \bold{A} \lVert \leq \epsilon$$
\end{block}
\begin{block}{The fixed rank problem}
Given $\bold{A}$ and $k$, seek $\bold{B}$ s. t.
$$\lVert \bold{A} - \bold{Q}\bold{Q}^T \bold{A} \lVert \approx \min_{rank(\bold{X}) \leq
k} \lVert \bold{A} - \bold{X} \lVert$$
\end{block}
}

\frame{
\frametitle{Stage A:}
\framesubtitle{The proto-algorithm:}
\begin{block}{The proto-algorithm}
1. Draw an n x (k + p) random matrix $\bold{O}$ \\
2. Form the matrix $\bold{Y} = \bold{A} \bold{O}$ \\
3. Construct a matrix $\bold{Q}$ whose columns form an othonormal basis for the
range of $\bold{Y}$ \\
\end{block}}

\frame{
\frametitle{Stage A:}
\framesubtitle{Randomized Range Finder:}
\begin{block}{Stage A - Randomized Range Finder}
1. Draw an n x l standard Gaussian matrix $\bold{O}$ \\
2. Form the m x l matrix $\bold{Y} = \bold{A} \bold{O}$ \\
3. Construct the m x l matrix $\bold{Q}$ using the QR factorization of:
$\bold{Y} = \bold{QR}$
\end{block}

The number of flops necessary for this algorithm is:
$$lnT_{rand}+lT_{mult}+l^2m$$
}

\frame{
\frametitle{Stage A:}
\framesubtitle{Randomized Power Iteration:}
\begin{block}{Randomized Power Iteration:}
1. Draw an n x l standard Gaussian matrix $\bold{O}$ \\
2. For the m x l matrix $\bold{Y} = (\bold{AA}^*)^q \bold{A} \bold{O}$ via alternative
application of $\bold{A}$ and $\bold{A}^q$ \\
3. Construct the m x l matrix $\bold{Q}$ using the QR decomposition:
$\bold{Y} = \bold{QR}$
\end{block}
}

\frame{
\frametitle{Stage A:}
\framesubtitle{Fast Randomized Range Finder:}
\begin{block}{Fast Randomized Range Finder}
1. Draw an n x l SRFT matrix $\bold{O}$ \\
2. Form the m x l matrix $\bold{Y} = \bold{A} \bold{O}$ \\
3. Construct the m x l matrix $\bold{Q}$ using the QR factorization of:
$\bold{Y} = \bold{QR}$
\end{block}
An SFRT is :
$$ \bold{O} = \sqrt{\frac{n}{l}}\bold{DFR}$$

where:
 $\bold{D}$ is a n x n diagonal matrix whose entries are random variables,
distributed on the complex unit circle
 $\bold{F}$ is the n x n unitary
discrete Fourier transform and 
 $\bold{R}$ is an n x l matrix whose $l$ column
are drawn from the n x n identity matrix.
}

\frame{
\frametitle{Stage B:}
\framesubtitle{Direct SVD:}
\begin{block}{Direct SVD}
1. Form the matrix $\bold{B} = \bold{Q}^* \bold{A}$ \\
2. Compute the SVD of the matrix $ \bold{B} = \tilde{\bold{U}} \bold{S} \bold{V}^*$ \\
3. Form the orthonormal matrix $\bold{U} = \bold{Q} \tilde{\bold{U}}$ \\
\end{block}
}

\frame{
\frametitle{Theory:}
\framesubtitle{Proto-Algorithm Error bound:}
We use the notations:
$$A=U \begin{pmatrix}
\Sigma_1 & \\
 & \Sigma_2
\end{pmatrix}
\begin{pmatrix}
V_1^*\\
V_2^*
\end{pmatrix}
$$
and
$$\Omega_1=V_1^*\Omega\ and \  \Omega_2=V_2^*\Omega$$

\begin{block}{Theorem:}
Assuming that $\Omega_1$ has full row rank, 

\begin{equation}
||(I-P_Y)A||^2\leq ||\Sigma_2||^2+||\Sigma_2\Omega_2\Omega_1^{\dagger}||
\end{equation}
\end{block}

}

\frame{
\frametitle{Theory:}
\framesubtitle{Error bound on the randomized range finder:}
\begin{block}{Theorem:}
We keep the same notations as before. $k,p\geq 2$ and $k+p\leq min(m,n)$:
\begin{equation}
\mathbb{E}||(I-P_Y)A||_F\leq (1+\frac{k}{p-1})^{1/2} (\sum_{j>k} \sigma_j^2)^{1/2}
\end{equation}

and for the spectral norm :

\begin{equation}
\mathbb{E}||(I-P_Y)A||\leq (1+\frac{k}{p-1})\sigma_{k+1}+ \frac{e\sqrt{k+p}}{p} (\sum_{j>k} \sigma_j^2)^{1/2}
\end{equation}
\end{block}

}

\frame{
\frametitle{Numericals:}
\framesubtitle{Image processing application:}
\begin{figure}[h]
   \includegraphics[scale=.25]{eigenvalue.png}
\end{figure}
 - The random range finder(blue) runs for 4.5 sec.\\
 - The random power iteration(green) runs for 9.8 sec.\\
 - The fast random range finder(red) runs for 11 sec.

}

\frame{
\frametitle{Conclusion:}
}


\end{document}
